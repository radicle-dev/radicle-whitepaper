\section{VCS Integration}

One of the fundamental goals of \oscoin{} is to build an open platform that
works for everyone. A key aspect of this is how the protocol integrates VCSs.
If \oscoin{} is to remain relevant in the future, it has to be designed in a
VCS-agnostic way. Thisw means that the core protocol should not be built
with any assumptions about the kind of VCS that will be used for a given project.

To achieve this, we build a `common core' which operates with patches, and a
set of adpaters for different VCSs.

\subsection{The Common Core}

Distributed VCS software today can be divided into two categories: \emph{patch}-
based and \emph{snapshot}-based. Examples of the former include darcs~\ref{darcs}
and examples of the latter include git~\ref{git} and mercurial~\ref{mercurial}.
The common core we build is based on the work by Samuel Mimram and Cinzia Di
Giusto in A Categorical Theory of Patches~\ref{categorical-theory-of-patches} --
a formulation of patch theory~\ref{patch-theory} which gives us a reliable
foundation on which to build our common core.

It's important that we be able to support both types of DVCSs, and patch theory
is stricly more powerful than snapshot based systems. This means we can support
snapshot-based systems with a patch-based one, but not the other way around.

\subsection{Adapters}

The adapters for VCSs perform one crucial operation: they translate the native
VCS information into the common core format when pushing changes upstream, and
do the inverse when pulling changes downstream.

For \texttt{git}, this takes the form of a \emph{git-remote-helper}~
\ref{git-remote-helper}.  When a \texttt{git} command is issued with an address
belonging to the \texttt{oscoin://} protocol scheme, \texttt{git} calls into
our remote helper instead of using the default one. This means git users
do not need to change their workflow when using \oscoin{}, they simply need
to point their client to a different remote.

% TODO: Push and then update chain?
% TODO: Client signs the tx with the push? Or push and tx are separate?
% TODO: Local blockchain?
% TODO: Is pushing decoupled from updating the chain?
% TODO: Merges/conflicts?
