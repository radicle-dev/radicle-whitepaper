\section{Introduction}
\label{s:introduction}

Blockchains are being used with increasing frequency, and for increasingly
diverse purposes. This had led to scaling concerns, as clients must keep
abreast of a large number of transactions, many of which are will not be
directly relevant to them. There are several protocol-level efforts to assuage
these concerns. The \oscoin{} architecture, however, emphasizes an additional,
\textit{semantic} level of sharding. Rather than a single chain, \oscoin{}
allows for an indefinite number of different chains, each serving different
purposes -- such as keeping track of a particular repository, of the outcome of
a vote, or of global financial transactions. In general, participants need not
sync chains which are not of interest to them, saving considerable space,
computation, and network usage.

Since the purpose of chains may thus be quite specific, we would like to be
able to precisely specify what sorts of transactions are allowed on it, and to
optimize syntax and semantics for those use-cases. In essence, we want
\textit{domain-specific languages for domain-specific chains}. The language
described in this paper, \rad, is designed to easily allow the definition of
such domain-specific languages (DSLs). Moreover, the very same mechanism used to
facilitate such definitions - the notion of \textit{reflective towers} - also
allows for redefining the semantics of a single chain, enabling a more
principled and first-class version of the self-amendment pioneered by
Tezos.\cite{Goodman2014}

In addition to being a language for on-chain programming, \rad is also a
scripting language that can be used to script local interactions with the
chain. Indeed, chains, scripts, and REPL interactions are conceived of
uniformly. This in turn allows quite novel new possibilities; one can for
example describe on one chain how a server should respond to updates of another
chain, so that redeployments to that server become as simple as submitting new
transactions to the server-chain.

\subsection{Context}

\subsection{Desiderata}

In order for consensus to be achievable, there must be agreement among
participants on the meaning and validity of blocks and transactions. Seen from
the perspective of programming languages, the meaning of an expression or
declaration must be the same for all users. This leads naturally to a
\textit{pure} programming language - one which side-effects are disallowed.

The chains on


\begin{figure}[H]
\begin{tabular}{| l | l | l |}
\hline
Version Control & Blockchain     & Language  \\ \hline
    Patch/Diff  & Transaction    & Declaration\footnote{\rad makes no
    distinctions between declarations and expressions, and so for the purposes
    of this paper one may consider this cell to be filled with ``Expression''}\\
Branch/Fork     & Fork           & ...  \\
Snapshot        & State          & Memory + Env \\
\hline
\end{tabular}
\label{correspondences}
\caption{Correspondences ...}
\end{figure}
